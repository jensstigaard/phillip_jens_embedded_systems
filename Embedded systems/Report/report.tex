\documentclass[12pt, danish, a4paper, titlepage]{article}

\usepackage[utf8]{inputenc}
\usepackage[danish]{babel}
\usepackage{color}
\usepackage[pdftex]{graphicx}
\usepackage{float}
\usepackage[top=2cm, bottom=3cm]{geometry}

\title{{\huge Assignment 1}\\02131 Embedded systems}
\author{
		Jens Grønhøj Stigaard 	(s113412)\\
		Phillip Barth 		(s113404)
}
\date{\small \today}

\begin{document}
	\maketitle	
	
	\tableofcontents
	\newpage	
	
	\section{Introduction}
	This rapport is part of the first assignment, in the DTU course 02131 Embedded systems.
	The full assignment was to create a fully functional heart monitor. This first part focuses on the program code, that is required for this heart monitor to work.
	
	\noindent This rapport will begin by analysing the requirements of a heart monitoring program. Thereafter we will take a look at the design choices, that we have taken, to full fill the requirements. 
	We will also take a look at some of the more interesting implementations of these designs.
	The results of the implementation will then be analysed and possible improvements will be discussed. 
	\newpage
	\section{Analysis}
	An in depth analysis of assignment:
		\subsection{User Requirements}
		The program is taking an numeric input, handles the data automatically and displays the important variables. It warns the user when critical data is retreived. The input data the program retreives has a specific interval, which makes possible to calculate the pulse of the patient, and not just the size of the input. 
		\subsection{Functions}
			\subsubsection{DataStream}
				In this first assignment, we do have input data as a file, which means what it's not a live datastream. 
			\subsubsection{Filters}
			\subsubsection{Peakanalysis}
			\subsubsection{Output}
			
	\section{Design}
	In this section, we would like to go in depth with design of the program. We will explain how the different parts of the code interact with each other, and give a general sense, for what these parts do and how they do it.
		\subsection{Overall Design}
		\subsection{Filters}
		\subsection{Peakdetection and Analysis}
		\subsection{Output}
			The main purpose of the ECG is to display the patient's data in a customized way that make sense for anyone who looks at the display. Furthermore, it has to be showed if the input seems to be invalid - like if there's no pulse. These cases has to be treated by the program.
		
	\section{Implementation}
	This next part will go in depth with some of the more interesting parts of the code itself. How we implemented the different design decisions.  
		\subsection{Filters}
		\subsection{Peakanalysis}
		
	\section{Results}
	This section will take a look at our resulting data, in comparison to the data provided, and will try to analyse this data and the data gathered by our performance analysis. 
		\subsection{Analysis}
		\subsection{Runtime}
		
	\section{Improvements}
	A couple of important improvements that could be made, given time.
		\subsection{Array Shifting}
		\subsection{Searchback improvement}
		\subsection{Runtime Analysis}
		
	\section{Discussion}
	Discussing the ramifications of the results and how the improvements could impact future results: 
	
	\section{Conclusion}
	In the end we can conclude that...
		
	
\end{document}
